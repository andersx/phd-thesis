\chapter{Conclusion and Outlook}

During the project described in this thesis and the attached papers, I have implemented a method to determine the structure of several small proteins using their experimental chemical shifts.
The structure of the CI-2 protein was solved in using only computer resources that are available in any lab, with only NMR data that was automatically recorded and assigned.

Lastly, I have attempted to fold several protein structures around 200 amino acids.
Out of 7 proteins greater than 100 residues, a good structure was located in 4 cases.
The last three likely failed due to inefficient use of the NOE restraints.
Since the existing code to handle NOE restraints in PHAISTOS did not perform well on large structures, I implemented a new NOE energy term, and this was used to fold the Rhodopsin structure (225 amino acids) to a CA-RMSD of 2.5 \AA~from the experimental X-ray structure using a set only 195 NOE data restraints and assigned backbone chemical shifts.
The same code was able to fold the Savinase structure to a CA-RMSD of 2.9 \AA~from the experimental X-ray structure using only distance restraints derived from evolutionary data and assigned chemical shifts.

This required implementing a version of CamShift, from scratch, in PHAISTOS, and implemented useful energy function rigorously founded in Bayesian statistics.
To aid the setup of calculations, a graphical user interface for PHAISTOS was created.
I have parametrized and implemented a version of ProCS to calculate amide proton chemical shifts, and shown that this parametrization yields structure that are in better agreement with experimental data than simulations using a chemical shift predictor parametrized from experimental data.
Furthermore, I have parametrized parts of the backbone atom ProCS chemical shift predictor and implemented this in a PHAISTOS module.
This required the implementation of FragBuilder Python API which was used to automatically setup, run, and collect data from more than 2,000,000 QM calculations.
