\chapter*{Dansk Resumé}
\addcontentsline{toc}{chapter}{Dansk Resumé (Danish Summary)}

Kemien af et protein er tæt forbundet med dens tre-dimensionelle struktur. Af denne grund, er proteinstruktur bestemmelse grundlaget for rationel forståelse af kemien af biologiske processer, der involverer proteiner.

For tiden er flest kendte proteinstrukturer blevet løst ved røntgenkrystallografi. Kravet til løsning af en struktur på denne måde er, at proteinet krystalliserer. Moderne krystaliserings-metoder dog kun har en succesrate på 5\% \cite{xray}. I disse tilfælde kan kernemagnetisk resonans (NMR) metoder anvendes med en vis succes.
I øjeblikket indeholder Protein Data Bank 90.000 strukturer løst ved røntgen- og 9.000 strukturer løst ved NMR-metoder, og omkring 10.000 røntgen- og 500 NMR-strukturer bliver indsendt hvert år \cite{PDB}.

Konventionelle NMR-metoder til bestemmelse af protein strukturer optager et flerdimensionelt spektrum, som korrelerer resonansfrekvenser flere kerner på samme tid.
Fra dette spektrum er først problem at tilordne de kemiske skift af hver kerne. Denne proces er i vid udstrækning automatiseret for hovedkædeatomer, men er mere involveret for sidekædeatomer.  Disse  oplysninger bruges til at identificere toppe i spektret, der svarer til afstandsbegrænsninger (NOE begrænsninger) mellem par af atomer. Disse distance begrænsninger er det bruges til at generere ensembler af strukturer, der tilfredsstiller det givne sæt af begrænsninger.
Protein NMR-spektroskopi har imidlertid flere begrænsninger. Store proteiner har meget overfyldte spektre , hvilket komplicerer opgaven - hovedsagelig på grund af brede toppe og resulterende spektraloverlapning.
Dette er en væsentlig hindring for tilordningen af de kemiske skift og dermed for at finde de
værdifuld NOE begrænsninger. Følgeligt har omkring 95 \% af alle NMR- strukturer i PDB-databasen således har en størrelse på kun 200 aminosyrer eller mindre. Dette kan sammenlignes med de gennemsnitlige størrelser af proteiner i mennesker og \textit{E. coli}, som er henholdsvis omkring 400-600 og 200-400. Problemet kan mindskes ved deuterering  som imidlertid falder til nummer NOE-begrænsninger, der kan findes. Isotopmærkningsmetoder som selektivt mærker visse sidekæder er blevet udviklet som en effektiv strategi for sådanne problemer.

\section*{Computerberegningsmetoder}
En anden tilgang til at løse en proteinstruktur fra aminosyresekvensen er simulering
af energilandskabet af proteinet. Dette kaldes også proteinfoldning. I denne
tilgang, er de mulige konformationer samplet og scoret med en beskrivelse af proteinernes fysik, uden ekstra viden fra eksperimenter. Sådanne \textit{ab initio} tilgange har været anvendet til at bestemme strukturer, typisk med en præcision ned til 3 \AA, via Monte Carlo simuleringer i ROSETTA-programmet \cite{rosetta}. Et andet næveværdigt eksempel er den samtidige bestemmelse af struktur
og dynamik flere små proteiner via meget lange molekylær dynamik (MD) simuleringer med
Anton computer \cite{rdcensemble}.

Selv om disse metoder ikke kræver noget eksperimentelt arbejde, er det ekstremt krævende i
forhold til de edb-ressourcer, der er nødvendige. Desuden er de normalt ikke nemme at konvergere for systemer $>100$ aminosyrer \cite{Lange2012}.
ROSETTA-metoden er (i øjeblikket) velsagtens den mest succesfulde metode til at bestemme
en proteinstruktur via computer beregninger. For nylig viste Baker gruppen, at optagelsen af hovedkæde kemiske skift og RDC data forbedrer ROSETTA-protokollen og tillader bestemmelse af strukturer op til 150 rester \cite{Baker2010,Lange2012}. 

Grundlaget for ROSETTA er fragment-samling af lokale proteinstrukturmodeller, kombineret med raffinering ved hjælp af en energifunktion, der er blevet påvist at fungere bemærkelsesværdigt godt. 
Kort beskrevet  består fuldatom-ROSETTA-energifunktion af flere additive temer som Lennard-Jones potentialer, termer for eksponering solvent, hydrogenbindinger, elektrostatiske par-interaktioner og dispersion-iteraktioner, og endelig torsions potentialer for hovedkæde- og sidekædevinkler. 

Nøjagtigheden af energifunktionen kommer dog på bekostning af beregningsmæssige hastighed og ufuldstændig i den konformationelle prøvetagning, som synes at være den uoverkommelige forhinding for yderligere succes for ROSETTA.
Denne protokol er for nylig blevet forbedret yderligere med inddragelse af meget sparsomme mængder NOE-data \cite{LangePNAS2012}.

Dette gav 7 strukturer omkring 200 aminosyrer, der blev bestemt med en nøjagtighed på mellem 2,5 og 3,9 \AA~fra de tilsvarende eksperimentelle røntgen-strukturer, og desuden blev en god struktur for det 376 aminosyrer store maltosebindingsprotein endda fundet, men dette krævede væsentligt flere NOE oplysninger. Disse simuleringer krævede en 512-kerner supercomputer som kørte i flere dage, for hvert protein.

Et andet nævneværdigt eksempel på protein strukturbestemmelse metoder, der beskæftiger NMR-data, er er CHESHIRE-metoden \cite{cheshire}. CHESHIRE-metoden var den første metode som løste strukturer kun ved brug af kemiske skift, og bruger en fragmentsamlingstilgang, efterfulgt af en Monte Carlo raffinering ved hjælp af et all-atom kraft-felt og en energi-funktion, der inkluderer kemiske skift. Denne metode blev anvendt til at bestemme proteinstrukturer fra kemiske skift, og fandt strukturer for 11 proteiner mellem 54 og 123 aminosyrer i størrelse, til en nøjagtighed på omkring 1,5 \AA~fra de tilsvarende eksperimentelle røntgen-strukturer.
\\\\I det følgende afsnit, er PHAISTOS-programmet introduceret, og formalisme for inkludering af kemiske skift i simuleringer i PHAISTOS er udledt. Dette er et forsøg på at løse
to centrale udfordringer i proteinfoldning: (1) Fuldstændig konformationel prøveudtagning og (2) nøjagtig energi-scoring af konformationelle prøver.
Disse udfordringer er mødt som følger: (1) ved hjælp af en nyudviklet forudindtaget konformationel prøveudtagningsmetode og (2) ved at parametrisere en nøjagtig kemisk skift forudsigelsesmetode, brut med en energifunktion baseret på Bayesiansk statistik, som tillader, at dette kombineres med eksisterende energifunktioner i PHAISTOS.
Denne kombinerede fremgangsmåde vil blive demonstreret på foldningssimuleringer på et testsæt af proteiner med kendte strukturer spænder fra 55 til 269 rester.
