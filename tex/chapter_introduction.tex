\chapter*{Preface}
\addcontentsline{toc}{chapter}{Preface}

Nuclear magnetic resonance (NMR) spectra of proteins are increasingly used in protein chemistry to obtain knowledge about both structure and function of proteins.

Conventionally, chemical shifts are measured and assigned in order to get the valuable nuclear Overhausen effect (NOE) and residual dipolar couplings (RDC) restraints used to determine a protein structure. NOE and RDC values relate directly to distances and relative orientations in the protein structure, and thus serve as a very powerful tool to determine the right structure.

The connection between chemical shifts and the protein structure is less straight-forward.
The chemical shift depends on the shielding of an external magnetic field by the electron density around a nucleus.
In other words, the wave function (and its derivatives with respect to the induced current and the nuclear magnetic moment) for a given protein structure must be known in order to know the related set of chemical shifts.
The fact that there is no clear geometric interpretation of chemical shifts makes it difficult to use as restraints to determine a protein structure.

It is, however, well-known, that chemical shifts correlate with both local secondary structure as well as non-local structure. For instance, H$^\alpha$ chemical shifts are typically larger in an $\alpha$-helices and smaller in $\beta$-sheets.

Fortunately, a multitude of approximations exists, which allow the chemical shifts of a protein to be calculated with high accuracy on the time scale of milliseconds (compared several days for a gas-phase quantum mechanical calculation).



