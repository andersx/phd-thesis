\chapter{Mass-spectroscopy in protein structure determination}

This section introduces the two experimental methods, hydrogen exchange mass spectroscopy (HXMS) and cross-correlation mass spectroscopy (XCMS), and possible application in protein folding.

These experimental methods are attractive, since the experiments are relatively easy to carry out.

mass spectroscopy 

\section{Cross-linking mass spectroscopy (XLMS)}


XLMS has previously been used to 

We considered several linkers of different lengths. The \textit{de facto} standard linkers, DSS and DST which measure 11 \aa ngstr\"o m and 6.4 \aa ngstr\"o m, respectively.

\begin{figure}
    \centering
    \includegraphics[width=0.65\textwidth]{figures/xcms/lognormal.pdf}
    \caption{linkers}
    \label{fig:linkers}
\end{figure}


\begin{figure}
    \centering
    \includegraphics[width=0.85\textwidth]{figures/hGH_rainbow.pdf}
    \caption{linkers}
    \label{fig:hGH_homology}
\end{figure}

\section{Hydrogen exchange mass spectroscopy (HXMS)}

\subsection{Hydrogen bond criterion}

The interresidue hydrogen bond criterion of the DSSP program\cite{dssp} is used to identify hydrogen bonds.
The DSSP program uses an electrostatic model, assuming partial charges of -0.42 e and +0.20 e to the carbonyl oxygen and amide hydrogen respectively, and -0.42 e and +0.20 e to the carbonyl carbon and amide nitrogen, respectively.
A hydrogen bond is empirically defined as having an interaction energy given as
\begin{equation}
E_\mathrm{HB} = \left(\frac{1}{r_\mathrm{ON}} + \frac{1}{r_\mathrm{OH}} - \frac{1}{r_\mathrm{CH}} - \frac{1}{r_\mathrm{CN}} \right)\cdot\ 27.89\ \mathrm{kcal/mol}
\end{equation}
stronger than a cut-off of -0.5 kcal/mol.

\subsection{Beta-binomial model}
The likelihood model that correlates a measured deuterium uptake to an integer number of hydrogen bonds in a strand is a simple model based on a beta-binomial distribution.

\begin{equation}
P(N_\mathrm{HB} = k | \alpha, \beta) = \binom{n}{k} \frac{B(k+\alpha, n - k + \beta)}{B(\alpha, \beta)}
\end{equation}
where $n$ is the length of the strand (i.e. the maximum possible number of hydrogen bonds), $N_\mathrm{HB}$ is the number of observed hydrogen bonds and $k \in {0, ... ,n}$ are the possible values of $N_\mathrm{HB}$. $\alpha$ and $\beta$ are variables and $B(x,y)$ is the beta-function given as:
\begin{equation}
B(x,y)=\frac{(x-1)!(y-1)!}{(x+y-1)!}
\end{equation}
The mean, $\mu$, and variance, $\sigma^2$, and variance of the beta-binomial distribution is
\begin{eqnarray}
    \mu      & = & \frac{n\alpha}{\alpha + \beta}\\
    \sigma^2 & = & \frac{n\alpha\beta(n + \alpha + \beta)}{(\alpha + \beta)^2(1 + \alpha + \beta)}
\end{eqnarray}
If $\mu$ and $\sigma^2$ are estimated, then $\alpha$ and $\beta$ can then be derived for a strand of length $n$ as:
\begin{eqnarray}
    \alpha & = & -\frac{\mu(\mu^2 - \mu n + \sigma^2)}{\sigma^2 n + \mu^2 - \mu n}\\
    \beta  & = & \frac{n\alpha}{\mu} - \alpha
\end{eqnarray}


\subsection{HXMS likelihood function}

The simplified forward-model that correlates the protein structure, $\mathbf{X}$ to the measured values of deuterium uptake, $\{D_i\}$ is obtained by constructing an empirical function $f(D_i) \mapsto \mu$











