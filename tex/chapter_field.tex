
\chapter{Introduction to the field}

Nuclear magnetic resonance (NMR) spectra of proteins are increasingly used in protein chemistry to obtain knowledge about both structure and function of proteins.

Conventionally, chemical shifts are measured and assigned in order to get the valuable nuclear Overhausen effect (NOE) and residual dipolar couplings (RDC) restraints used to determine a protein structure. NOE and RDC values relate directly to distances and relative orientations in the protein structure, and thus serve as a very powerful tool to determine the right structure.

The connection between chemical shifts and the protein structure is less straight-forward.
The chemical shift depends on the shielding of an external magnetic field by the electron density around a nucleus.
In other words, the wave function (and its derivatives with respect to the induced current and the nuclear magnetic moment) for a given protein structure must be known in order to know the related set of chemical shifts.
Fortunately, a multitude of approximations exists, which allow the chemical shifts of a protein to be calculated with high accuracy on the time scale of milliseconds (compared several days for a gas-phase quantum mechanical calculation).

The fact that there is no clear geometric interpretation of chemical shifts makes it difficult to use as restraints to determine a protein structure.
It is, however, well-known, that chemical shifts correlate with both local secondary structure as well as non-local structure. For instance, H$^\alpha$ chemical shifts are typically larger in an $\alpha$-helices and smaller in $\beta$-sheets, and H$^\mathrm{N}$ that engage in short hydrogen bonds typically have large chemical shifts than if they were in a longer hydrogen bond.

Typically chemical shifts are used in protein structure determination in two different ways.
(1) via an energy function that scores the agreement between predicted chemical shifts and experimental structures or (2) via a biased introduced at the conformation sampling stage.

Usually these approaches are employ Monte Carlo sampling, although Vendruscolo and co-workers have explored a chemical shift biased molecular dynamics approach.


A number of methods to use chemical shifts in protein folding have been proposed. This section describes some of the most notable approaches.

\section{ROSETTA}

The ROSETTA methodology is (currently) arguably the most successful method to determine a protein structure computationally.
The basis of ROSETTA is an energy function that has been demonstrated to work remarkably. Briefly described, the all-atom ROSETTA energy function consists of several additive terms such as Lennard-Jones potentials, terms for solvent exposure, hydrogen bonding, electrostatic pair-interactions and dispersion interactions, and finally torsional potentials for backbone and side chain angles.
The weights between the terms are empirically optimized.

The strength of the ROSETTA energy function is that in nearly all reported cases, the experimental X-ray structure has a lower energy than any other proposed structure. The demonstrated accuracy of the energy function does come at the cost of computational speed and incomplete conformational sampling seems to be the prohibitive for further success for ROSETTA. Consequently, most publication using ROSETTA employ hundreds to thousands of cores running for several days in order to determine one structure.

Recently, the ROSETTA method has been extended to include various forms of NMR data. Chemical shifts are used to bias the fragments from which all-atom protein structures are constructed, which is the minimized through one of ROSETTA's protocols.

The largest structures determined by ROSETTA are summed up in Tab 

\section{CHESHIRE}

\section{Vendruscolo CS-MD}

\section{Meiler and Baker}

\section{Ad Bax}

\section{Conventional // CYANA, etc}

\section{\textit{Ab initio} methods}


