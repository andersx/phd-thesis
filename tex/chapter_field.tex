\chapter{Problems}

NMR based protein structure prediction has several obstacles that prevent NMR from being the go-to method in many situations.





\chapter{Exiting Methods}
A number of methods to use chemical shifts in protein folding have been proposed. This section describes some of the most notable approaches.

\section{ROSETTA}

The ROSETTA methodology is (currently) arguably the most successful method to determine a protein structure computationally.
The basis of ROSETTA is an energy function that has been demonstrated to work remarkably. Briefly described, the all-atom ROSETTA energy function consists of several additive terms such as Lennard-Jones potentials, terms for solvent exposure, hydrogen bonding, electrostatic pair-interactions and dispersion interactions, and finally torsional potentials for backbone and side chain angles.
The weights between the terms are empirically optimized.

The strength of the ROSETTA energy function is that in nearly all reported cases, the experimental X-ray structure has a lower energy than any other proposed structure. The demonstrated accuracy of the energy function does come at the cost of computational speed and incomplete conformational sampling seems to be the prohibitive for further success for ROSETTA. Consequently, most publication using ROSETTA employ hundreds to thousands of cores running for several days in order to determine one structure.

Recently, the ROSETTA method has been extended to include various forms of NMR data. Chemical shifts are used to bias the fragments from which all-atom protein structures are constructed, which is the minimized through one of ROSETTA's protocols.

The largest structures determined by ROSETTA are summed up in Tab 

\section{CHESHIRE}

\section{Vendruscolo CS-MD}

\section{Meiler and Baker}

\section{Ad Bax}

\section{Conventional // CYANA, etc}

\section{\textit{Ab initio} methods}


